\documentclass[letterpaper,11pt]{article}
\usepackage[hyphens]{url} 
\usepackage{latexsym}
\usepackage[empty]{fullpage}
\usepackage{titlesec}
\usepackage{marvosym}
\usepackage[usenames,dvipsnames]{color}
\usepackage{verbatim}
\usepackage{enumitem}
\usepackage[english]{babel}
\usepackage{tabularx}
\usepackage{fontawesome5}
\usepackage{multicol}
\usepackage{graphicx}
\usepackage{CormorantGaramond}
\usepackage{csquotes}
\usepackage{charter} 

\setlength{\multicolsep}{-3.0pt}
\setlength{\columnsep}{-1pt}
\input{glyphtounicode}

\setlength{\itemsep}{-1pt}

\RequirePackage{tikz}
\RequirePackage{xcolor}

\definecolor{cvblue}{RGB}{32, 64, 151}
\definecolor{black}{HTML}{130810}
\definecolor{darkcolor}{HTML}{0F4539}
\definecolor{cvgreen}{HTML}{3BD80D}
\definecolor{taggreen}{HTML}{00E278}
\definecolor{SlateGrey}{HTML}{2E2E2E}
\definecolor{LightGrey}{HTML}{666666}
\colorlet{name}{black}
\colorlet{tagline}{darkcolor}
\colorlet{heading}{darkcolor}
\colorlet{headingrule}{cvblue}
\colorlet{accent}{darkcolor}
\colorlet{emphasis}{SlateGrey}
\colorlet{body}{LightGrey}

% Adjust margins
\addtolength{\oddsidemargin}{-0.6in}
\addtolength{\evensidemargin}{-0.5in}
\addtolength{\textwidth}{1.19in}
\addtolength{\topmargin}{-.7in}
\addtolength{\textheight}{1.4in}
\urlstyle{same}

\definecolor{airforceblue}{rgb}{0.36, 0.54, 0.66}

\raggedbottom
\raggedright
\setlength{\tabcolsep}{0in}

% Sections formatting
\titleformat{\section}{
  \vspace{-4pt}\scshape\raggedright\large\bfseries
}{}{0em}{}[\color{black}\titlerule \vspace{-5pt}]

% Ensure that generate pdf is machine readable/ATS parsable
\pdfgentounicode=1

%-------------------------
% Custom commands
\newcommand{\resumeItem}[1]{
  \item\small{
    {#1 \vspace{-1pt}}
  }
}

\newcommand{\classesList}[4]{
    \item\small{
        {#1 #2 #3 #4 \vspace{-2pt}}
  }
}

\newcommand{\resumeSubheading}[4]{
  \vspace{-2pt}\item
    \begin{tabular*}{1.0\textwidth}[t]{l@{\extracolsep{\fill}}r}
      \textbf{\large#1} & \textbf{\small #2} \\
      \textit{\large#3} & \textit{\small #4} \\
      
    \end{tabular*}\vspace{-7pt}
}


\newcommand{\resumeSingleSubheading}[4]{
  \vspace{-2pt}\item
    \begin{tabular*}{1.0\textwidth}[t]{l@{\extracolsep{\fill}}r}
      \textbf{\large#1} & \textbf{\small #2} \\
      
    \end{tabular*}\vspace{-7pt}
}

\newcommand{\resumeSubSubheading}[2]{
    \item
    \begin{tabular*}{0.97\textwidth}{l@{\extracolsep{\fill}}r}
      \textit{\small#1} & \textit{\small #2} \\
    \end{tabular*}\vspace{-7pt}
}


\newcommand{\resumeProjectHeading}[2]{
    \item
    \begin{tabular*}{1.001\textwidth}{l@{\extracolsep{\fill}}r}
      \small#1 & \textbf{\small #2}\\
    \end{tabular*}\vspace{-7pt}
}

\newcommand{\resumeSubItem}[1]{\resumeItem{#1}\vspace{-4pt}}

\renewcommand\labelitemi{$\vcenter{\hbox{\tiny$\bullet$}}$}
\renewcommand\labelitemii{$\vcenter{\hbox{\tiny$\bullet$}}$}

\newcommand{\resumeSubHeadingListStart}{\begin{itemize}[leftmargin=0.0in, label={}]}
\newcommand{\resumeSubHeadingListEnd}{\end{itemize}}
\newcommand{\resumeItemListStart}{\begin{itemize}[leftmargin=0.1in]}
\newcommand{\resumeItemListEnd}{\end{itemize}\vspace{-5pt}}

\newcommand\sbullet[1][.5]{\mathbin{\vcenter{\hbox{\scalebox{#1}{$\bullet$}}}}}

\newcommand{\RomanNumeralCaps}[1]{\MakeUppercase{\romannumeral #1}}
%-------------------------------------------\usepackage{fontspec}
%%%%%%  RESUME STARTS HERE  %%%%%%%%%%%%%%%%%%%%%%%%%%%%

\begin{document}

%%%%%%% ----------------------------------
%%%%%%%  HEADER
%%%%%%% ----------------------------------
\begin{center}
    \begin{minipage}[b]{0.24\textwidth}
%            \large Fairbanks, AK \\
            \large (208) 705-2666 
            % 
    \end{minipage}% 
    \begin{minipage}[b]{0.45\textwidth}
            \centering
            {\Huge Dempsey Rogers} \\ %
            \vspace{0.1cm}
            {\color{cvblue} \Large{Physicist, Mathematician, Machine Learning Scientist}} \\
    \end{minipage}% 
    \begin{minipage}[t]{0.24\textwidth}
           \flushright 
            \vspace{-.19in}

            \large{Dempsey.Rogers@gmail.com}

    \end{minipage}   

\vspace{-0.15cm} 
{\color{cvblue} \hrulefill}
\end{center}
\vspace{-0.2cm}
Highly driven physicist with 3 years of experience modeling turbulent transport dynamics in tokamak plasmas, and two prior years building and studying Naiver-Stokes and magnetohydrodynamic models. Research and Development oriented ML Scientist leveraging physics informed, unsupervized, and supervized machine learning models. 

\vspace{-0.2cm}

%%%%%%% --------------------------------
%%%%%%%  EDUCATION
%%%%%%% --------------------------------

\section*{\color{cvblue}{Education} }
\subsection*{{\color{cvblue}Master of Science, Physics,} {University of Alaska Fairbanks} \hfill 08/2018 --- 07/2021} 

\begin{itemize}
    \setlength{\itemsep}{-.5pt}
    \item Thesis: Phase Effects on the Turbulent Transport in the Magnetic Confinement of Plasmas for Nuclear Fusion
\end{itemize}
\subsection*{{\color{cvblue}Dual Bachelors of Arts, Mathematics and Physics,} {Carroll College} \hfill 08/2012 --- 05/2016} 
\begin{itemize}
    \setlength{\itemsep}{-.5pt}
    \item Senior Thesis: 2D Computational Study of MHD Instabilities for Nuclear Fusion Energy
\end{itemize}



%%%%%%% ----------------------------------
\section*{\color{cvblue}Skills}
\vspace{-.25in}
\begin{table}[h!]
    \setlength{\tabcolsep}{8pt}
    \renewcommand{\arraystretch}{2}
    \centering
    \begin{tabular}{p{0.12\linewidth} || p{0.8\linewidth}}
        Physics& Nonlinear Dynamics, Geophysical Fluid Dynamics, Mathematical Physics, Classical Mechanics,
        Quantum Mechanics, Computational Methods, Electricity and Magnetism, and Computational Plasma Physics RA \\\hline
        Mathematics& Applied Optimization, Nonlinear and Partial Differential Equations, Numerical Methods, Statistics, Discrete Mathematics, Real and Complex Analysis, Abstract Algebra and Non-Euclidean Geometry \\\hline
        Machine Learning & Research and Development: Iterated, documented, and deployed HSA, NN, CNN, and LSTM models leveraging Python, PyTorch, PSQL, Splunk, BASH, Slurm, MATLAB, \LaTeX,  Beamer, and the Microsoft Suite \\\hline
        Other& WFR, CPR, First Aid, and Winter Wilderness Survival\\\hline
    \end{tabular}
    % \caption{Caption}
    % \label{tab:my_label}
\end{table}



%%%%%%% -----------------------------------------
%%%%%%%  Work Experience 
%%%%%%% ---------------------------------
\section*{\color{cvblue}Work Experience } 

\subsection*{{\color{cvblue}Data Scientist, Idaho National Laboratory } \hfill 05/2023 — Current} 
\begin{itemize}
    \setlength{\itemsep}{-.5pt}
    \item  Research and develop new technologies and applications for anomaly detection through machine learning, including deep learning architectures and novel methods
    \item  Keep abreast of state of the art technologies and techniques for modeling, analyzing, manipulating, and storing data
    \item  Provide imaginative, thorough machine learning solutions to a wide range of complex, ambiguous, and difficult problems related to protecting American critical infrastructure
    \item  Develop and produce deep learning models related to autonomous nuclear fission reactor controls
    \item  Clearances: DOE Q 08/2022, DHS/CISA Suitability 07/2023, and DOE SCI 10/2023
\end{itemize}


\subsection*{{\color{cvblue}Classification Analyst, Idaho National Laboratory } \hfill 02/2022 — 05/2023} 
 
\begin{itemize}
    \setlength{\itemsep}{-.5pt}
    \item  Utilize engineering theory and principles in the development, interpretation, and administration of approved Department of Energy (DOE) Classification, Controlled Unclassified Information Programs, applicable Information
    Security Department Programs, and working groups
    \item  Provide guidance on classification and sensitive information matters to INL Derivative Classifiers. Provide classification and declassification reviews for INL and Idaho Cleanup Project potentially classified materials
    \item  Apply DOE classification principles and concepts to assure that documents within potentially classified subject areas are reviewed for classified and sensitive information and are properly protected
    \item  Develop a machine learning model for Natural Language Processing (NLP) to aid in the classification of documents
    \item  Research state of the art NLPs, build question answering and summarizing models, and develop and fine-tune models using the INL’s High Performance Computing facility
    \item  Generate question and answering data sets to fine tune models towards classification vernacular
\end{itemize}

\subsection*{{\color{cvblue}Computational Physics RA., UAF } \hfill 12/2019 --- 12/2021} 
    \begin{itemize}
    \setlength{\itemsep}{-.5pt}
    \item Applied skills in mathematics, statistics, and physics on UAF's High Performance Computer (HPC) to study nuclear fusion devices  
    \item Investigated a proposed thermal transport control mechanism for tokamak plasmas in the I-mode confinement regime 
    \item Computationally modeled the turbulent transport dynamics from phase effects in the thermal transport of tokamak plasmas
    \item Verified model fidelity by modeling observed transport dynamics from D-\RomanNumeralCaps{3}D
    \item Investigated hysteresis free control mechanisms to increase core ion temperature and decrease core density 
    \item Compiled and publicly defended a master's thesis based on findings
\end{itemize}

%%%%%%% -----------------------------------

\subsection*{{\color{cvblue} Physics Lab TA., UAF } \hfill 08/2018 --- 12/2019} 
\begin{itemize}
    \setlength{\itemsep}{-.5pt}
    \item Prepared experiments demonstrating key concepts from the course lectures
    \item Provided short lectures at the being of lab sections to highlight concepts demonstrated in the experiment
    \item Developed distance learning labs, materials, as well as interactive remote test prep, homework, and lab support for students
    \item Assisted in course projects, daily assignments, and illustrated parallels between course work and real world applications
\end{itemize}

\section*{\color{cvblue}References} 


\subsection*{{\color{cvblue}Ashley Shields,} Idaho National Laboratory\hfill 12/2024 --- Current}
Computational Data Science, Senior Manager
\begin{itemize}
    \setlength{\itemsep}{-.5pt}
    \item (208) 526-5479
    \item Ashley.Shields@inl.gov
\end{itemize}
 
\subsection*{{\color{cvblue}John Koudelka,} Idaho National Laboratory\hfill 05/2023 --- 12/2024}
Scientific Visualization, Senior Manager
\begin{itemize}
    \setlength{\itemsep}{-.5pt}
    \item (208) 526-8591
    \item John.Koudelka@inl.gov
\end{itemize}
 
\subsection*{{\color{cvblue}Neil Walker,} Idaho National Laboratory\hfill 02/2022 --- 05/2023}
Information Security, Senior Manager
\begin{itemize}
    \setlength{\itemsep}{-.5pt}
    \item (208) 526-5641
    \item Neil.Walker@inl.gov
\end{itemize}

\end{document}